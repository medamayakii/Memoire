\documentclass{ujarticle}
\usepackage{sotsuron}
\pagestyle{myheadings}
\twocolumn
\begin{document}
\section{まとめ}

本論文では,ノイズファジィクラスタリングの概念を導入することで,ANFISによる分類器をノイズのあるクラスラベルに対するロバスト化を目的として改良した.
各個体の非ノイズ度をFCM法のようにメンバシップ値で表し,これをANFISのモデル推定に利用してノイズの影響を除去する.
提案法の特徴を,数値実験において従来法と比較すると,テストデータの分類性能が向上することが示された.
また,ANFISの前提部のメンバシップ関数を比較すると,従来のモデルではノイズ個体に過適合し,尖った形状となった一方,提案法では,ノイズの多い訓練データに対してもなめらかな関数を維持することができた.

今後の課題としては,クラス比が不均衡であったり,クラスの重複があるような他種類のデータに対するノイズ感度を調査することが挙げられる.

\section*{謝辞}

本研究は本学工学域電気電子系学類情報工学課程の本多克宏教授,生方誠希准教授の御指導のもとに行われたものであり,心より感謝の意を表します.

\end{document}
