\documentclass[fleqn]{ujarticle}
\usepackage{sotsuron}
\pagestyle{myheadings}
\twocolumn
\begin{document}

\section{はじめに}
Adaptive Network-based Fuzzy Inference System (ANFIS)~\cite{Jang1993}は,高木-菅野ファジィ推論システム(TS-FIS)~\cite{Takagi1985}を実現するための人工ニューラルネットワークアーキテクチャであり,ファジィIf-Thenルールベースを利用した説明可能なニューラルネットワークの有望なモデルである.
ANFISは,5層の人工ニューラルネットワークアーキテクチャを用いてTS-FISの推論を実現しており,その前提部パラメータは,逆伝播における勾配降下法により更新され,結論部パラメータは,順伝播における最小二乗推定によって更新される.
数値予測や2値分類のための多入力1出力(MISO)のモデルに加え,多クラス分類のための多入力多出力(MIMO)のモデルにも適用することができる~\cite{Angelov2004}.
以下本論文では,主に多クラス分類のためのMIMO-ANFISを取り扱う.

最小二乗型の目的関数は,ノイズに影響されやすい性質があるため,多くの最小二乗型モデルでは,ノイズ除去が大きな課題となる.
ANFISにおいてもノイズの多いデータセットに用いる際には,ノイズ除去を組み込んで設計する必要がある.
MISOの場合においては,ANFISにおいて2種類のノイズ処理を行うことができる~\cite{Honda_ICIEV2021}.
入力レベルのノイズは,ノイズ個体がすべてのIf-Thenルールに対して小さいファジィメンバシップ値をもつように前提部のメンバシップ関数をロバストに推定することによって除去する.
一方,出力レベルのノイズは,モデルの出力がノイズ出力に過適合しないように結論部の回帰係数をロバストに推定することによって除去する.
本論文では,多クラス分類において「ノイズクラスラベル」を取り扱う.
分類器の設計の目標は,出来る限り正確に訓練データのクラスラベルを再構築することであるが,教師となるクラスラベルがノイズに汚染されていると,その汎化性能は著しく低下してしまうため,ノイズラベルにロバストな分類器を慎重に設計する必要がある.

ノイズファジィクラスタリングは,ファジィクラスタリングのロバスト化手法である.
一般的にfuzzy $c$-means(FCM)~\cite{Bezdek81a,Miyamoto_book08}のようなファジィクラスタリング手法では,クラスタリング基準がノイズや外れ値の影響を受けやすい最小二乗基準で定義されることが多い.
元のFCM法の過程におけるノイズ個体や外れ個体の影響を除去するため,Davé~\cite{Dave91,Dave97}は,すべての個体と等しい距離をとると仮定した追加のノイズクラスタを導入し,すべての正常なクラスタから離れたノイズ個体や外れ個体をそこに帰属するようにした.
ロバストなモデル推定の観点から,ノイズファジィクラスタリングは反復的に重み付けされた最小二乗法~\cite{Holland77}の一種ととらえることができ,統計的な仮定を用いないノイズ除去手法として,robust PCA-based $k$-means~\cite{Honda2010_IEEE-TFS}のようなロバストなデータ分析で利用されている.

本論文では,ノイズクラスタリング手法をANFISの多クラス分類モデルに導入する.
この分類モデルでは,ANFISのモデル構築において正しくラベル付けされた個体は大きい重みを持ち,間違ったラベルをもつノイズ個体は無視されるように,各個体の非ノイズ度を結論部で推定する.
そして,ノイズ個体が追加のノイズクラスタに吸収されるようにファジィメンバシップ値を利用し,ANFISの分類器をロバストに構築する.
メンバシップ関数によって重み付けした最小二乗法による目的関数の最小化と非ノイズファジィメンバシップ値の推定を収束するまで繰り返す.
提案法の特徴を,実世界のデータセットを用いた数値実験により示す.

以下,本論文は,次のように構成されている.
第2章では,ANFISに基づく分類器とノイズファジィクラスタリングを導入したロバストなモデル構築について概説する.
第3章では,新しいロバストなANFIS分類器を提案する.
第4章では,実験結果を示し,第5章では,結論を述べる.

\end{document}
